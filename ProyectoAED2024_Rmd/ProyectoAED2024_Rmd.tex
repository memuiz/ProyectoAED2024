%  LaTeX support: latex@mdpi.com
%  For support, please attach all files needed for compiling as well as the log file, and specify your operating system, LaTeX version, and LaTeX editor.

%=================================================================
% pandoc conditionals added to preserve backwards compatibility with previous versions of rticles

\documentclass[notspecified,article,submit,moreauthors,pdftex]{Definitions/mdpi}


%% Some pieces required from the pandoc template
\setlist[itemize]{leftmargin=*,labelsep=5.8mm}
\setlist[enumerate]{leftmargin=*,labelsep=4.9mm}


%--------------------
% Class Options:
%--------------------

%---------
% article
%---------
% The default type of manuscript is "article", but can be replaced by:
% abstract, addendum, article, book, bookreview, briefreport, casereport, comment, commentary, communication, conferenceproceedings, correction, conferencereport, entry, expressionofconcern, extendedabstract, datadescriptor, editorial, essay, erratum, hypothesis, interestingimage, obituary, opinion, projectreport, reply, retraction, review, perspective, protocol, shortnote, studyprotocol, systematicreview, supfile, technicalnote, viewpoint, guidelines, registeredreport, tutorial
% supfile = supplementary materials

%----------
% submit
%----------
% The class option "submit" will be changed to "accept" by the Editorial Office when the paper is accepted. This will only make changes to the frontpage (e.g., the logo of the journal will get visible), the headings, and the copyright information. Also, line numbering will be removed. Journal info and pagination for accepted papers will also be assigned by the Editorial Office.

%------------------
% moreauthors
%------------------
% If there is only one author the class option oneauthor should be used. Otherwise use the class option moreauthors.

%---------
% pdftex
%---------
% The option pdftex is for use with pdfLaTeX. Remove "pdftex" for (1) compiling with LaTeX & dvi2pdf (if eps figures are used) or for (2) compiling with XeLaTeX.

%=================================================================
% MDPI internal commands - do not modify
\firstpage{1}
\makeatletter
\setcounter{page}{\@firstpage}
\makeatother
\pubvolume{1}
\issuenum{1}
\articlenumber{0}
\pubyear{2023}
\copyrightyear{2023}
%\externaleditor{Academic Editor: Firstname Lastname}
\datereceived{ }
\daterevised{ } % Comment out if no revised date
\dateaccepted{ }
\datepublished{ }
%\datecorrected{} % For corrected papers: "Corrected: XXX" date in the original paper.
%\dateretracted{} % For corrected papers: "Retracted: XXX" date in the original paper.
\hreflink{https://doi.org/} % If needed use \linebreak
%\doinum{}
%\pdfoutput=1 % Uncommented for upload to arXiv.org

%=================================================================
% Add packages and commands here. The following packages are loaded in our class file: fontenc, inputenc, calc, indentfirst, fancyhdr, graphicx, epstopdf, lastpage, ifthen, float, amsmath, amssymb, lineno, setspace, enumitem, mathpazo, booktabs, titlesec, etoolbox, tabto, xcolor, colortbl, soul, multirow, microtype, tikz, totcount, changepage, attrib, upgreek, array, tabularx, pbox, ragged2e, tocloft, marginnote, marginfix, enotez, amsthm, natbib, hyperref, cleveref, scrextend, url, geometry, newfloat, caption, draftwatermark, seqsplit
% cleveref: load \crefname definitions after \begin{document}

%=================================================================
% Please use the following mathematics environments: Theorem, Lemma, Corollary, Proposition, Characterization, Property, Problem, Example, ExamplesandDefinitions, Hypothesis, Remark, Definition, Notation, Assumption
%% For proofs, please use the proof environment (the amsthm package is loaded by the MDPI class).

%=================================================================
% Full title of the paper (Capitalized)
\Title{Título}

% MDPI internal command: Title for citation in the left column
\TitleCitation{Título}

% Author Orchid ID: enter ID or remove command
%\newcommand{\orcidauthorA}{0000-0000-0000-000X} % Add \orcidA{} behind the author's name
%\newcommand{\orcidauthorB}{0000-0000-0000-000X} % Add \orcidB{} behind the author's name


% Authors, for the paper (add full first names)
\Author{Alejandro León Líndez$^{1}$, Adrian Lizzadro Pla$^{2}$, Marta
Medina Muñiz$^{3}$}


%\longauthorlist{yes}


% MDPI internal command: Authors, for metadata in PDF
\AuthorNames{Alejandro León Líndez, Adrian Lizzadro Pla, Marta Medina
Muñiz}

% MDPI internal command: Authors, for citation in the left column

% Affiliations / Addresses (Add [1] after \address if there is only one affiliation.)
\address{%
$^{1}$ \quad Máster en Ciencia de
Datos; \href{mailto:alelin@alumni.uv.es}{\nolinkurl{alelin@alumni.uv.es}}\\
$^{2}$ \quad Máster en Ciencia de
Datos; \href{mailto:alizpla@alumni.uv.es}{\nolinkurl{alizpla@alumni.uv.es}}\\
$^{3}$ \quad Máster en Ciencia de
Datos; \href{mailto:memuiz@alumni.uv.es}{\nolinkurl{memuiz@alumni.uv.es}}\\
}

% Contact information of the corresponding author
\corres{Correspondence: }

% Current address and/or shared authorship








% The commands \thirdnote{} till \eighthnote{} are available for further notes

% Simple summary
\simplesumm{Resumen.}

%\conference{} % An extended version of a conference paper

% Abstract (Do not insert blank lines, i.e. \\)
\abstract{Abstract}


% Keywords
\keyword{keyword 1; keyword 2; keyword 3 (list three to ten pertinent
keywords specific to the article, yet reasonably common within the
subject discipline.).}

% The fields PACS, MSC, and JEL may be left empty or commented out if not applicable
%\PACS{J0101}
%\MSC{}
%\JEL{}

%%%%%%%%%%%%%%%%%%%%%%%%%%%%%%%%%%%%%%%%%%
% Only for the journal Diversity
%\LSID{\url{http://}}

%%%%%%%%%%%%%%%%%%%%%%%%%%%%%%%%%%%%%%%%%%
% Only for the journal Applied Sciences

%%%%%%%%%%%%%%%%%%%%%%%%%%%%%%%%%%%%%%%%%%

%%%%%%%%%%%%%%%%%%%%%%%%%%%%%%%%%%%%%%%%%%
% Only for the journal Data



%%%%%%%%%%%%%%%%%%%%%%%%%%%%%%%%%%%%%%%%%%
% Only for the journal Toxins


%%%%%%%%%%%%%%%%%%%%%%%%%%%%%%%%%%%%%%%%%%
% Only for the journal Encyclopedia


%%%%%%%%%%%%%%%%%%%%%%%%%%%%%%%%%%%%%%%%%%
% Only for the journal Advances in Respiratory Medicine
%\addhighlights{yes}
%\renewcommand{\addhighlights}{%

%\noindent This is an obligatory section in “Advances in Respiratory Medicine”, whose goal is to increase the discoverability and readability of the article via search engines and other scholars. Highlights should not be a copy of the abstract, but a simple text allowing the reader to quickly and simplified find out what the article is about and what can be cited from it. Each of these parts should be devoted up to 2~bullet points.\vspace{3pt}\\
%\textbf{What are the main findings?}
% \begin{itemize}[labelsep=2.5mm,topsep=-3pt]
% \item First bullet.
% \item Second bullet.
% \end{itemize}\vspace{3pt}
%\textbf{What is the implication of the main finding?}
% \begin{itemize}[labelsep=2.5mm,topsep=-3pt]
% \item First bullet.
% \item Second bullet.
% \end{itemize}
%}


%%%%%%%%%%%%%%%%%%%%%%%%%%%%%%%%%%%%%%%%%%

% Pandoc syntax highlighting
\usepackage{color}
\usepackage{fancyvrb}
\newcommand{\VerbBar}{|}
\newcommand{\VERB}{\Verb[commandchars=\\\{\}]}
\DefineVerbatimEnvironment{Highlighting}{Verbatim}{commandchars=\\\{\}}
% Add ',fontsize=\small' for more characters per line
\usepackage{framed}
\definecolor{shadecolor}{RGB}{248,248,248}
\newenvironment{Shaded}{\begin{snugshade}}{\end{snugshade}}
\newcommand{\AlertTok}[1]{\textcolor[rgb]{0.94,0.16,0.16}{#1}}
\newcommand{\AnnotationTok}[1]{\textcolor[rgb]{0.56,0.35,0.01}{\textbf{\textit{#1}}}}
\newcommand{\AttributeTok}[1]{\textcolor[rgb]{0.13,0.29,0.53}{#1}}
\newcommand{\BaseNTok}[1]{\textcolor[rgb]{0.00,0.00,0.81}{#1}}
\newcommand{\BuiltInTok}[1]{#1}
\newcommand{\CharTok}[1]{\textcolor[rgb]{0.31,0.60,0.02}{#1}}
\newcommand{\CommentTok}[1]{\textcolor[rgb]{0.56,0.35,0.01}{\textit{#1}}}
\newcommand{\CommentVarTok}[1]{\textcolor[rgb]{0.56,0.35,0.01}{\textbf{\textit{#1}}}}
\newcommand{\ConstantTok}[1]{\textcolor[rgb]{0.56,0.35,0.01}{#1}}
\newcommand{\ControlFlowTok}[1]{\textcolor[rgb]{0.13,0.29,0.53}{\textbf{#1}}}
\newcommand{\DataTypeTok}[1]{\textcolor[rgb]{0.13,0.29,0.53}{#1}}
\newcommand{\DecValTok}[1]{\textcolor[rgb]{0.00,0.00,0.81}{#1}}
\newcommand{\DocumentationTok}[1]{\textcolor[rgb]{0.56,0.35,0.01}{\textbf{\textit{#1}}}}
\newcommand{\ErrorTok}[1]{\textcolor[rgb]{0.64,0.00,0.00}{\textbf{#1}}}
\newcommand{\ExtensionTok}[1]{#1}
\newcommand{\FloatTok}[1]{\textcolor[rgb]{0.00,0.00,0.81}{#1}}
\newcommand{\FunctionTok}[1]{\textcolor[rgb]{0.13,0.29,0.53}{\textbf{#1}}}
\newcommand{\ImportTok}[1]{#1}
\newcommand{\InformationTok}[1]{\textcolor[rgb]{0.56,0.35,0.01}{\textbf{\textit{#1}}}}
\newcommand{\KeywordTok}[1]{\textcolor[rgb]{0.13,0.29,0.53}{\textbf{#1}}}
\newcommand{\NormalTok}[1]{#1}
\newcommand{\OperatorTok}[1]{\textcolor[rgb]{0.81,0.36,0.00}{\textbf{#1}}}
\newcommand{\OtherTok}[1]{\textcolor[rgb]{0.56,0.35,0.01}{#1}}
\newcommand{\PreprocessorTok}[1]{\textcolor[rgb]{0.56,0.35,0.01}{\textit{#1}}}
\newcommand{\RegionMarkerTok}[1]{#1}
\newcommand{\SpecialCharTok}[1]{\textcolor[rgb]{0.81,0.36,0.00}{\textbf{#1}}}
\newcommand{\SpecialStringTok}[1]{\textcolor[rgb]{0.31,0.60,0.02}{#1}}
\newcommand{\StringTok}[1]{\textcolor[rgb]{0.31,0.60,0.02}{#1}}
\newcommand{\VariableTok}[1]{\textcolor[rgb]{0.00,0.00,0.00}{#1}}
\newcommand{\VerbatimStringTok}[1]{\textcolor[rgb]{0.31,0.60,0.02}{#1}}
\newcommand{\WarningTok}[1]{\textcolor[rgb]{0.56,0.35,0.01}{\textbf{\textit{#1}}}}

% tightlist command for lists without linebreak
\providecommand{\tightlist}{%
  \setlength{\itemsep}{0pt}\setlength{\parskip}{0pt}}




\begin{document}



%%%%%%%%%%%%%%%%%%%%%%%%%%%%%%%%%%%%%%%%%%

\section{Introducción}\label{introducciuxf3n}

\section{Carga de librerías e importación del
fichero}\label{carga-de-libreruxedas-e-importaciuxf3n-del-fichero}

\begin{Shaded}
\begin{Highlighting}[]
\FunctionTok{rm}\NormalTok{(}\AttributeTok{list=}\FunctionTok{ls}\NormalTok{())  }\CommentTok{\# Borrar todas las variables al principio}
\end{Highlighting}
\end{Shaded}

\begin{Shaded}
\begin{Highlighting}[]
\FunctionTok{library}\NormalTok{(readr)}
\FunctionTok{library}\NormalTok{(ggplot2)}
\end{Highlighting}
\end{Shaded}

\begin{Shaded}
\begin{Highlighting}[]
\NormalTok{viajes }\OtherTok{\textless{}{-}} \FunctionTok{read\_delim}\NormalTok{(}\StringTok{"data/viajeros\_según\_sexo\_destino\_duracion\_del\_viaje.csv"}\NormalTok{, }\AttributeTok{delim =} \StringTok{";"}\NormalTok{, }\AttributeTok{escape\_double =} \ConstantTok{FALSE}\NormalTok{, }\AttributeTok{trim\_ws =} \ConstantTok{TRUE}\NormalTok{)}
\end{Highlighting}
\end{Shaded}

\begin{verbatim}
## Rows: 648 Columns: 6
## -- Column specification --------------------------------------------------------
## Delimiter: ";"
## chr (5): Sexo, Duración, Destino, Tipo de dato, Total
## dbl (1): Periodo
## 
## i Use `spec()` to retrieve the full column specification for this data.
## i Specify the column types or set `show_col_types = FALSE` to quiet this message.
\end{verbatim}

\begin{Shaded}
\begin{Highlighting}[]
\FunctionTok{any}\NormalTok{(}\FunctionTok{is.na}\NormalTok{(viajes))  }\CommentTok{\# Comprobación de si hay datos faltantes}
\end{Highlighting}
\end{Shaded}

\begin{verbatim}
## [1] FALSE
\end{verbatim}

\subsection{Transformar a tidy data}\label{transformar-a-tidy-data}

\begin{Shaded}
\begin{Highlighting}[]
\NormalTok{tipos\_datos }\OtherTok{\textless{}{-}} \FunctionTok{unique}\NormalTok{(viajes}\SpecialCharTok{$}\StringTok{\textasciigrave{}}\AttributeTok{Tipo de dato}\StringTok{\textasciigrave{}}\NormalTok{)}
\end{Highlighting}
\end{Shaded}

\begin{Shaded}
\begin{Highlighting}[]
\NormalTok{viajes\_porcentaje\_categoria }\OtherTok{\textless{}{-}} \FunctionTok{subset}\NormalTok{(viajes,viajes}\SpecialCharTok{$}\StringTok{\textasciigrave{}}\AttributeTok{Tipo de dato}\StringTok{\textasciigrave{}} \SpecialCharTok{==}\NormalTok{ tipos\_datos[}\DecValTok{1}\NormalTok{])}
\NormalTok{nombres\_col\_cat }\OtherTok{\textless{}{-}} \FunctionTok{colnames}\NormalTok{(viajes\_porcentaje\_categoria)}
\NormalTok{nombres\_col\_cat[nombres\_col\_cat }\SpecialCharTok{==} \StringTok{"Total"}\NormalTok{] }\OtherTok{\textless{}{-}}\StringTok{"Porcentaje\_personas\_categoria"}
\FunctionTok{colnames}\NormalTok{(viajes\_porcentaje\_categoria)}\OtherTok{\textless{}{-}}\NormalTok{nombres\_col\_cat}
\NormalTok{viajes\_porcentaje\_categoria }\OtherTok{\textless{}{-}} \FunctionTok{subset}\NormalTok{(viajes\_porcentaje\_categoria, }\AttributeTok{select =} \SpecialCharTok{{-}}\StringTok{\textasciigrave{}}\AttributeTok{Tipo de dato}\StringTok{\textasciigrave{}}\NormalTok{)}
\end{Highlighting}
\end{Shaded}

\begin{Shaded}
\begin{Highlighting}[]
\NormalTok{viajes\_porcentaje\_total}\OtherTok{\textless{}{-}} \FunctionTok{subset}\NormalTok{(viajes,viajes}\SpecialCharTok{$}\StringTok{\textasciigrave{}}\AttributeTok{Tipo de dato}\StringTok{\textasciigrave{}} \SpecialCharTok{==}\NormalTok{ tipos\_datos[}\DecValTok{2}\NormalTok{])}
\NormalTok{nombres\_col\_tot }\OtherTok{\textless{}{-}} \FunctionTok{colnames}\NormalTok{(viajes\_porcentaje\_total)}
\NormalTok{nombres\_col\_tot[nombres\_col\_tot }\SpecialCharTok{==} \StringTok{"Total"}\NormalTok{] }\OtherTok{\textless{}{-}}\StringTok{"Porcentaje\_personas\_total"}
\FunctionTok{colnames}\NormalTok{(viajes\_porcentaje\_total)}\OtherTok{\textless{}{-}}\NormalTok{nombres\_col\_tot}
\NormalTok{viajes\_porcentaje\_total }\OtherTok{\textless{}{-}} \FunctionTok{subset}\NormalTok{(viajes\_porcentaje\_total, }\AttributeTok{select =} \SpecialCharTok{{-}}\StringTok{\textasciigrave{}}\AttributeTok{Tipo de dato}\StringTok{\textasciigrave{}}\NormalTok{)}
\end{Highlighting}
\end{Shaded}

\begin{Shaded}
\begin{Highlighting}[]
\CommentTok{\# Compruebo que en todas las columnas salvo la última todos las filas son iguales para poder hacer merge de los dos datasets correctamente}
\FunctionTok{any}\NormalTok{(viajes\_porcentaje\_categoria[}\DecValTok{1}\SpecialCharTok{:}\FunctionTok{length}\NormalTok{(}\FunctionTok{nrow}\NormalTok{(viajes\_porcentaje\_categoria)}\SpecialCharTok{{-}}\DecValTok{1}\NormalTok{)] }\SpecialCharTok{!=}\NormalTok{viajes\_porcentaje\_total[}\DecValTok{1}\SpecialCharTok{:}\FunctionTok{length}\NormalTok{(}\FunctionTok{nrow}\NormalTok{(viajes\_porcentaje\_total)}\SpecialCharTok{{-}}\DecValTok{1}\NormalTok{)])}
\end{Highlighting}
\end{Shaded}

\begin{verbatim}
## [1] FALSE
\end{verbatim}

\begin{Shaded}
\begin{Highlighting}[]
\CommentTok{\# Uno los dos datasets en un único dataset con el que trabajar}

\NormalTok{datos }\OtherTok{\textless{}{-}} \FunctionTok{merge}\NormalTok{(viajes\_porcentaje\_categoria,viajes\_porcentaje\_total, }\AttributeTok{by =} \FunctionTok{c}\NormalTok{(}\FunctionTok{colnames}\NormalTok{(viajes\_porcentaje\_categoria[}\DecValTok{1}\SpecialCharTok{:}\FunctionTok{length}\NormalTok{(viajes\_porcentaje\_categoria)}\SpecialCharTok{{-}}\DecValTok{1}\NormalTok{])))}
\end{Highlighting}
\end{Shaded}

\subsection{Transformación de clases}\label{transformaciuxf3n-de-clases}

\begin{Shaded}
\begin{Highlighting}[]
\FunctionTok{lapply}\NormalTok{(datos,class)}
\end{Highlighting}
\end{Shaded}

\begin{verbatim}
## $Sexo
## [1] "character"
## 
## $Duración
## [1] "character"
## 
## $Destino
## [1] "character"
## 
## $Periodo
## [1] "numeric"
## 
## $Porcentaje_personas_categoria
## [1] "character"
## 
## $Porcentaje_personas_total
## [1] "character"
\end{verbatim}

\begin{Shaded}
\begin{Highlighting}[]
\NormalTok{datos}\SpecialCharTok{$}\NormalTok{Porcentaje\_personas\_categoria }\OtherTok{\textless{}{-}}\FunctionTok{sub}\NormalTok{(}\StringTok{","}\NormalTok{,}\StringTok{"."}\NormalTok{,datos}\SpecialCharTok{$}\NormalTok{Porcentaje\_personas\_categoria)}

\CommentTok{\# Compruebo si se han introducido NA}
\FunctionTok{which}\NormalTok{(}\FunctionTok{is.na}\NormalTok{(}\FunctionTok{as.numeric}\NormalTok{(datos}\SpecialCharTok{$}\NormalTok{Porcentaje\_personas\_categoria))}\SpecialCharTok{==}\ConstantTok{TRUE}\NormalTok{)}
\end{Highlighting}
\end{Shaded}

\begin{verbatim}
## Warning in which(is.na(as.numeric(datos$Porcentaje_personas_categoria)) == :
## NAs introducidos por coerción
\end{verbatim}

\begin{verbatim}
## [1] 133
\end{verbatim}

\begin{Shaded}
\begin{Highlighting}[]
\CommentTok{\# Compruebo que dato habia antes de la transformación para poder corregirlo si fuera posible}
\NormalTok{datos}\SpecialCharTok{$}\NormalTok{Porcentaje\_personas\_categoria[}\DecValTok{133}\NormalTok{]}
\end{Highlighting}
\end{Shaded}

\begin{verbatim}
## [1] ".."
\end{verbatim}

\begin{Shaded}
\begin{Highlighting}[]
\CommentTok{\# No es posible corregirlo, es un dato faltante}
\end{Highlighting}
\end{Shaded}

\begin{Shaded}
\begin{Highlighting}[]
\NormalTok{datos}\SpecialCharTok{$}\NormalTok{Porcentaje\_personas\_categoria }\OtherTok{\textless{}{-}} \FunctionTok{as.numeric}\NormalTok{(datos}\SpecialCharTok{$}\NormalTok{Porcentaje\_personas\_categoria)}
\end{Highlighting}
\end{Shaded}

\begin{verbatim}
## Warning: NAs introducidos por coerción
\end{verbatim}

\begin{Shaded}
\begin{Highlighting}[]
\NormalTok{datos}\SpecialCharTok{$}\NormalTok{Porcentaje\_personas\_total}\OtherTok{\textless{}{-}}\FunctionTok{sub}\NormalTok{(}\StringTok{","}\NormalTok{,}\StringTok{"."}\NormalTok{,datos}\SpecialCharTok{$}\NormalTok{Porcentaje\_personas\_total)}

\CommentTok{\# Compruebo si se han introducido NA}
\FunctionTok{which}\NormalTok{(}\FunctionTok{is.na}\NormalTok{(}\FunctionTok{as.numeric}\NormalTok{(datos}\SpecialCharTok{$}\NormalTok{Porcentaje\_personas\_total))}\SpecialCharTok{==}\ConstantTok{TRUE}\NormalTok{)}
\end{Highlighting}
\end{Shaded}

\begin{verbatim}
## Warning in which(is.na(as.numeric(datos$Porcentaje_personas_total)) == TRUE):
## NAs introducidos por coerción
\end{verbatim}

\begin{verbatim}
## [1] 133
\end{verbatim}

\begin{Shaded}
\begin{Highlighting}[]
\CommentTok{\# Compruebo que dato habia antes de la transformación para poder corregirlo si fuera posible}
\NormalTok{datos}\SpecialCharTok{$}\NormalTok{Porcentaje\_personas\_total[}\DecValTok{133}\NormalTok{]}
\end{Highlighting}
\end{Shaded}

\begin{verbatim}
## [1] ".."
\end{verbatim}

\begin{Shaded}
\begin{Highlighting}[]
\CommentTok{\# No es posible corregirlo, es un dato faltante}
\end{Highlighting}
\end{Shaded}

No hay datos para mujeres que en 2021 se fueron de viaje al extranjero
de 1 a 3 noches.

\begin{Shaded}
\begin{Highlighting}[]
\NormalTok{datos}\SpecialCharTok{$}\NormalTok{Porcentaje\_personas\_total }\OtherTok{\textless{}{-}} \FunctionTok{as.numeric}\NormalTok{(datos}\SpecialCharTok{$}\NormalTok{Porcentaje\_personas\_total)}
\end{Highlighting}
\end{Shaded}

\begin{verbatim}
## Warning: NAs introducidos por coerción
\end{verbatim}

\begin{Shaded}
\begin{Highlighting}[]
\CommentTok{\# Comprobamos que se hayan transformado correctamente}
\FunctionTok{lapply}\NormalTok{(datos,class) }
\end{Highlighting}
\end{Shaded}

\begin{verbatim}
## $Sexo
## [1] "character"
## 
## $Duración
## [1] "character"
## 
## $Destino
## [1] "character"
## 
## $Periodo
## [1] "numeric"
## 
## $Porcentaje_personas_categoria
## [1] "numeric"
## 
## $Porcentaje_personas_total
## [1] "numeric"
\end{verbatim}

%%%%%%%%%%%%%%%%%%%%%%%%%%%%%%%%%%%%%%%%%%

\vspace{6pt}

%%%%%%%%%%%%%%%%%%%%%%%%%%%%%%%%%%%%%%%%%%
%% optional

% Only for the journal Methods and Protocols:
% If you wish to submit a video article, please do so with any other supplementary material.
% \supplementary{The following supporting information can be downloaded at: \linksupplementary{s1}, Figure S1: title; Table S1: title; Video S1: title. A supporting video article is available at doi: link.}

%%%%%%%%%%%%%%%%%%%%%%%%%%%%%%%%%%%%%%%%%%







%%%%%%%%%%%%%%%%%%%%%%%%%%%%%%%%%%%%%%%%%%
%% Optional

%% Only for journal Encyclopedia

\abbreviations{Abbreviations}{
The following abbreviations are used in this manuscript:\\

\noindent
\begin{tabular}{@{}ll}
MDPI & Multidisciplinary Digital Publishing Institute \\
DOAJ & Directory of open access journals \\
TLA & Three letter acronym \\
LD & linear dichroism \\
\end{tabular}}

%%%%%%%%%%%%%%%%%%%%%%%%%%%%%%%%%%%%%%%%%%
%% Optional
\input{"appendix.tex"}
%%%%%%%%%%%%%%%%%%%%%%%%%%%%%%%%%%%%%%%%%%
\begin{adjustwidth}{-\extralength}{0cm}

%\printendnotes[custom] % Un-comment to print a list of endnotes


\reftitle{References}
\bibliography{mybibfile.bib}

% If authors have biography, please use the format below
%\section*{Short Biography of Authors}
%\bio
%{\raisebox{-0.35cm}{\includegraphics[width=3.5cm,height=5.3cm,clip,keepaspectratio]{Definitions/author1.pdf}}}
%{\textbf{Firstname Lastname} Biography of first author}
%
%\bio
%{\raisebox{-0.35cm}{\includegraphics[width=3.5cm,height=5.3cm,clip,keepaspectratio]{Definitions/author2.jpg}}}
%{\textbf{Firstname Lastname} Biography of second author}

%%%%%%%%%%%%%%%%%%%%%%%%%%%%%%%%%%%%%%%%%%
%% for journal Sci
%\reviewreports{\\
%Reviewer 1 comments and authors’ response\\
%Reviewer 2 comments and authors’ response\\
%Reviewer 3 comments and authors’ response
%}
%%%%%%%%%%%%%%%%%%%%%%%%%%%%%%%%%%%%%%%%%%
\PublishersNote{}
\end{adjustwidth}


\end{document}
